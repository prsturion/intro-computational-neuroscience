\documentclass[portuguese,11pt,a4paper]{article}
\usepackage[T1]{fontenc}
\usepackage{graphicx}
\usepackage{mathtools}
\usepackage{amssymb}
\usepackage{amsthm}
\usepackage{thmtools}
\usepackage{xcolor}
\usepackage{nameref}
\usepackage{babel}
\usepackage{authblk}
\usepackage{fourier}
\usepackage{indentfirst}
\usepackage{float}
\usepackage{booktabs}
\usepackage[colorlinks, urlcolor=blue]{hyperref}
\title{Introdução à Neurociência Computacional\\Lista de Exercícios 8}
\author{Paulo R. Sturion}
\begin{document}
	\maketitle
	
	\noindent Todos os códigos escritos para produzir os resultados dos exercícios a seguir estão disponibilizados de forma clara e organizada no repositório Github:
	
	\begin{center}
		\noindent \href{https://github.com/prsturion/intro-computational-neurosciene.git}{https://github.com/prsturion/intro-computational-neuroscience.git} \newline
	\end{center}
	
	
	\noindent\textbf{Questão 1:}
	
	\noindent\textbf{(a)} 
	
	\begin{figure}[H]
		\centering
		\includegraphics[width=11cm]{../figures/ex_1a.png}
		\caption{Simulação do modelo de neurônios acoplados da Questão 1.}
	\end{figure}
	
	\noindent\textbf{(b)}
	
	\begin{figure}[H]
		\centering
		\includegraphics[width=11cm]{../figures/ex_1b.png}
		\caption{Simulação do modelo de neurônios acoplados da Questão 1 com a inversão de papeis pedida no item (b).}
	\end{figure}
	
	\noindent\textbf{(c)}
	
	\begin{figure}[H]
		\centering
		\includegraphics[width=11cm]{../figures/ex_1c.png}
		\caption{Resultados do item (c).}
	\end{figure}
	
	\noindent\textbf{(d)}
	
	\begin{figure}[H]
		\centering
		\includegraphics[width=8cm]{../figures/ex_1d_g_010.png}
	\end{figure}
	
	\begin{figure}[H]
		\centering
		\includegraphics[width=8cm]{../figures/ex_1d_g_012.png}
	\end{figure}
	
	\begin{figure}[H]
		\centering
		\includegraphics[width=8cm]{../figures/ex_1d_g_014.png}
	\end{figure}
	
	\begin{figure}[H]
		\centering
		\includegraphics[width=8cm]{../figures/ex_1d_g_016.png}
	\end{figure}
	
	\begin{figure}[H]
		\centering
		\includegraphics[width=8cm]{../figures/ex_1d_g_018.png}
	\end{figure}
	
	\begin{figure}[H]
		\centering
		\includegraphics[width=8cm]{../figures/ex_1d_g_020.png}
	\end{figure}
	
	\begin{figure}[H]
		\centering
		\includegraphics[width=8cm]{../figures/ex_1d_g_022.png}
	\end{figure}
	
	\begin{figure}[H]
		\centering
		\includegraphics[width=8cm]{../figures/ex_1d_g_024.png}
	\end{figure}
	
	\begin{figure}[H]
		\centering
		\includegraphics[width=8cm]{../figures/ex_1d_g_026.png}
	\end{figure}
	
	\begin{figure}[H]
		\centering
		\includegraphics[width=8cm]{../figures/ex_1d_g_028.png}
		\caption{Gráficos dos disparos e da fase para os diferentes valores de $\bar{g}_{EI}$.}
	\end{figure}
	

	Ao varrer $\bar g_{EI}\in\{0{,}10,0{,}12,0{,}14,0{,}16,0{,}18,0{,}20,0{,}22,0{,}24,0{,}26,0{,}28\}\ \text{mS/cm}^2$, observou-se:
	
	\begin{itemize}
		\item \textbf{Sincronização $1{:}1$ com travamento de fase:} 
		para $\bar g_{EI}=\{0{,}10,\ 0{,}12,\ 0{,}14\}\ \text{mS/cm}^2$, o neurônio I dispara um pico para cada pico de E, 
		com atraso aproximadamente constante $\delta\simeq 3{,}8\to 2{,}1\ \text{ms}$ ao longo dos disparos.
		\item \textbf{Acima de $\bar g_{EI}\ge 0{,}16\ \text{mS/cm}^2$:} 
		não há travamento de fase; o atraso $\delta$ varia sistematicamente (inclusive tornando-se negativo), 
		caracterizando perda de sincronização $1{:}1$.
		\item \textbf{Sincronização $n{:}1$:} não foi observada para nenhum dos valores testados 
		(isto é, não houve $n{:}1$ com $n\ge 2$).
	\end{itemize}
	
		
	
	
	
\end{document}